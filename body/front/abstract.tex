%------------------------ Abstract ------------------------

\begin{center}
\LARGE\textbf{چکیده}
\end{center}
\noindent
نگارش پایان‌نامه‌ علاوه بر بخش پژوهش و آماده‌سازی محتوا، مستلزم رعایت نکات فنی و نگارشی دقیقی است که در تهیه‌ی یک پایان‌نامه‌ی موفق بسیار کلیدی و موثر است. از آن جایی که بسیاری از نکات فنی مانند قالب کلی صفحات، شکل و اندازه‌ی قلم، صفحات عنوان و غیره در تهیه‌ی پایان‌نامه‌ها یکسان است، با استفاده از نرم‌افزار حروف‌چینی زی‌لاتک\LTRfootnote{\XeLaTeX} و افزونه‌ی زی‌پرشین\LTRfootnote{\XePersian} یک قالب استاندارد برای تهیه‌ی پایان‌نامه‌ها ارایه گردیده است. زی‌پرشین توسط دکتر وفا خلیقی توسعه داده شده است و در حال حاضر بهترین افزونه‌ی پارسی برای تهیه‌ی متون در لاتک\LTRfootnote{\LaTeX} به شمار می‌رود. این قالب می‌تواند برای تهیه‌ی پایان‌نامه‌های کارشناسی و کارشناسی ارشد و نیز رساله‌ها‌ی دکتری مورد استفاده قرار گیرد. این نوشتار به طور مختصر نحوه‌ی استفاده از این قالب را نشان می‌دهد. قالب اولیه توسط دکتر ضرابی‌زاده و دانشجویان دانشکده‌ی مهندسی کامپیوتر شریف تهیه شد که از {\href{https://github.com/zarrabi/thesis-template}{مخزن کد} آن در \lr{GitHub} قابل دریافت است. این نسخه توسط حسین رهنما ویرایش و شخصی‌سازی شده است. تغییرات عمده در این ویرایش حذف دستورات پارسی در کد لاتک، تغییر ساختار فایل‌ها و پوشه‌ها، بهبود کدها و بهبود ظاهر قالب است.

\bigskip
\noindent
\textbf{کلیدواژه‌ها.}
پایان‌نامه، حروف‌چینی، قالب استاندارد، لاتک، زی‌تک، زی‌پرشین.
\newpage
