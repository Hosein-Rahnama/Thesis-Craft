%------------------------ Preliminaries ------------------------

\chapter{مفاهیم اولیه}
دومین فصل پایان‌نامه به طور معمول به معرفی مفاهیمی می‌پردازد که در پایان‌نامه مورد استفاده قرار می‌گیرند. در این فصل به عنوان یک نمونه، نکات کلی در خصوص نحوه‌ی نگارش پایان‌نامه و نیز برخی نکات نگارشی به اختصار توضیح داده می‌شوند.


\section{نحوه‌ی نگارش}
\subsection{ساختار فایل‌ها و پوشه‌ها}
فایل‌ اصلی پایان‌نامه در این قالب \code{thesis.tex} نام دارد. به ازای هر فصل از پایان‌نامه، یک پوشه در شاخه‌ی \code{body/chapters-x} ایجاد شده است و نام آن در  \code{thesis.tex} (در قسمت فصل‌ها) درج گردیده است. برای مشاهده‌ی خروجی، فایل \code{thesis.tex} را با زی‌لاتک و بیب‌تک\LTRfootnote{BibTeX} کامپایل کنید. مشخصات اصلی پایان‌نامه به زبان پارسی را می‌توانید در پرونده‌ی \code{body/front/info.tex} ویرایش کنید. تمامی محتویاتی که در پایان‌نامه‌‌ی خود باید درج کنید در پوشه‌ی \code{body} قرار گرفته است. پوشه‌ی \code{styles} شامل پکیج‌های مورد استفاده و تنظمیمات این قالب است که در صورت نیاز می‌توانید آن را تغییر دهید. تمامی عکس‌ها در پوشه‌ی \code{figs} قرار دارند. هم‌چنین پوشه‌ی \code{fonts} دربرگیرنده‌ی فونت‌های مورد استفاده در این قالب است.


\subsection{عبارات ریاضی}
برای درج عبارات ریاضی در داخل متن از \texttt{\$...\$} و برای درج عبارات ریاضی در یک خط مجزا از \texttt{\$\$...\$\$ } یا محیط \code{equation} استفاده کنید. برای مثال عبارت $2x + 3y$ در داخل متن و عبارت زیر
\begin{equation}
\sum_{k=0}^{n} \binom{n}{k} = 2^n
\end{equation}
در یک خط مجزا درج شده است. دقت کنید که تمامی عبارات ریاضی، از جمله متغیرهای تک‌حرفی مانند $x$ و $y$ باید در محیط ریاضی یعنی محصور بین دو علامت \texttt{\$} باشند. 


\subsection{نمادهای ریاضی پرکاربرد}
برخی نمادهای ریاضی پرکاربرد در زیر فهرست شده‌اند. برای مشاهده‌ی دستور  معادل پرونده‌ی منبع را ببینید.
\begin{itemize}
\item
مجموعه‌‌های اعداد: 
$\mathbb{N}, \mathbb{Z}, \mathbb{Z}^+, \mathbb{Q}, \mathbb{R}, \mathbb{C}$
\item
مجموعه:
$\set{1, 2, 3}$
\item
دنباله‌:
$\seq{1, 2, 3}$
\item
سقف و کف:
$\ceil{x}, \floor{x}$
\item
اندازه و متمم:
$\card{A}, \setcomp{A}$
\item
همنهشتی:
$a \iequiv{n} 1$
یا
$a \equiv 1 \imod{n}$
\item
شمردن (عاد کردن):
$3 \divs n, 2 \ndivs n$
\item
ضرب و تقسیم:
$\times, \cdot, \div$
\item
سه‌نقطه‌:
$1, 2, \dots, n$
\item
کسر و ترکیب:
$\frac{n}{k}, \binom{n}{k}$
\item
اجتماع و اشتراک:
$A \cup (B \cap C)$
\item
عملگرهای منطقی:
$\neg p \vee (q \wedge r)$
\item
پیکان‌ها:
$\rightarrow, \Rightarrow, \leftarrow, \Leftarrow, \leftrightarrow, \Leftrightarrow$
\item
عملگرهای مقایسه‌ای:
$\not=, \le, \not\le, \ge, \not\ge$
\item
عملگرهای مجموعه‌ای:
$\in, \not\in, \setminus, \subset, \subseteq, \subsetneq, \supset, \supseteq, \supsetneq$
\item
جمع و ضرب چندتایی:
$\sum_{i=1}^{n} a_i, \prod_{i=1}^{n} a_i$
\item
اجتماع و اشتراک چندتایی:
$\bigcup_{i=1}^{n} A_i, \bigcap_{i=1}^{n} A_i$
\item
برخی نمادها:
$\infty, \varnothing, \forall, \exists, \triangle, \angle, \ell, \equiv, \therefore$
\end{itemize}


\subsection{لیست‌ها}
برای ایجاد یک لیست‌ می‌توانید از محیط‌های \code{itemize} و \code{enumerate} همانند زیر استفاده کنید.
\begin{multicols}{2}
\begin{itemize}
\item مورد اول
\item مورد دوم
\item مورد سوم
\end{itemize}

\begin{enumerate}
\item مورد اول
\item مورد دوم
\item مورد سوم
\end{enumerate}
\end{multicols}


\subsection{درج شکل}
یکی از روش‌های مناسب برای ایجاد شکل استفاده از نرم‌افزار \lr{LaTeX Draw} و سپس گرفتن خروجی آن به صورت یک فایل \code{tex}، کامپایل کردن آن به صورت یک فایل   \code{pdf} و قرار دادن آن درون متن با استفاده از محیط  \code{figure} است. همچنین می‌توانید با استفاده از نرم‌افزار \lr{Ipe} شکل‌های خود را مستقیما به صورت \code{pdf} ایجاد نموده و آن‌ها را به صورت عکس درون متن درج کنید. از بسته‌ی \code{TikZ} هم می‌توانید برای کشیدن طیف وسیعی از اشکال و به طور خاص گراف‌ها استفاده کنید. قرار دادن دو شکل در کنار هم به گونه‌ای که ترتیب آن‌ها در فهرست شکل‌ها حفظ شود کمی دردسر زاست و باید از روش‌های خاصی برای آن استفاه کرد. برای این منظور از محیط مخصوص \code{rowfig} استفاده کنید. برای نمونه، \cref{fig:graph} و \cref{fig:strip} با این روش کنار هم قرار گرفته‌اند.
\begin{rowfig}{t!}
\figbox{graph/graph.pdf}{5cm}{\caption{نمونه شکل ایجاد شده توسط نرم‌افزار \lr{LaTeX Draw}} \label{fig:graph}}
\sep
\figbox{strip.pdf}{8cm}{\caption{نمونه شکل ایجادشده توسط نرم‌افزار \lr{Ipe}} \label{fig:strip}}
\end{rowfig}				   
					   

\subsection{درج جدول}
برای درج جدول می‌توانید با استفاده از دستور \code{tabular} جدول را ایجاد کرده و سپس با دستور  \code{table}  آن را درون متن درج کنید. برای نمونه جدول \ref{tbl:operators} را ببینید.
\begin{table}[b!]
\centering
\caption{عملگرهای مقایسه‌ای}
\label{tbl:operators}
\begin{tabular}{|c|c|}
\hline
\textbf{عملگر} & \textbf{عنوان} \\ 
\hline \hline
\code{<} & کوچک‌تر \\ 
\code{>} & بزرگ‌تر \\
\code{==} & مساوی \\ 
\code{<>} & نامساوی \\ 
\hline
\end{tabular}
\end{table}


\subsection{درج الگوریتم}
برای درج الگوریتم می‌توانید از محیط \code{alg} استفاده کنید. یک نمونه در الگوریتم \ref{alg:cover:vertex} آمده است. همانطور که مشاهده می‌کنید کلمات کلیدی به رنگ آبی تیره و کامنت‌ها به رنگ سبز نمایش داده می‌شوند.

\subsection{محیط‌های ویژه}
برای درج مثال‌ها، قضیه‌ها، لم‌ها و نتیجه‌ها به ترتیب از محیط‌های \code{exmp}، \code{thm}، \code{lem} و \code{cor} استفاده کنید. برای درج اثبات قضیه‌ها و لم‌ها  از محیط \code{prf} استفاده کنید. تعریف‌های داخل متن را با استفاده از دستور به صورت \textbf{تیره‌} نشان دهید. تعریف‌های پایه‌ای‌تر را درون محیط \code{defn} قرار دهید.

\begin{alg}{t!}{پوشش راسی حریصانه}{alg:cover:vertex}
\AlgInput
گراف $G=(V, E)$
\AlgOutput
یک پوشش راسی از $G$
\vspace*{0.3em}
\Procedure{\lr{Spanning-Vertices}:}{}
\State
قرار بده $C = \varnothing$
\While {($E \ne \varnothing$)}
\If {($|E| > 10$)}
    \State 
یک کاری انجام بده
\ElsIf {($|E| \leq 5$)}
	\State
کار دیگری انجام بده
\Else
	\State
و یه کار دیگه!
\EndIf
\State
یال دل‌‌خواه $(u, v) \in E$ را انتخاب کن
\State
تمام یال‌های واقع بر $u$ یا $v$ را از $E$ حذف کن
\LineComment{این یک کامنت است.}
\State
راس‌های $u$ و $v$ را به $C$ اضافه کن
\EndWhile
\State \Return $C$
\EndProcedure
\end{alg}

\begin{defn}[اصل لانه‌ی کبوتری]
اگر $n+1$ کبوتر یا بیش‌تر درون  $n$ لانه قرار گیرند، آن‌گاه لانه‌ای 
وجود دارد که شامل حداقل دو کبوتر است.
\end{defn}
در هندسه‌ی مسطحه قضیه‌ی زیر را داریم.
\begin{thm}[فیثاغورس]
اگر اعداد حقیقی مثبت $a < b < c$ سه ضلغ یک مثلث قایم‌الزاویه باشند، آن‌گاه $c^2 = a^2 + b^2$.
\end{thm}

\begin{rem}
در گزاره‌ی بالا اگر دو ضلع کوچک‌تر مثلث برابر باشند یعنی $a = b$، آن‌گاه داریم $c = \sqrt{2}a$.
\end{rem}


\section{برخی نکات نگارشی}
این فصل حاوی برخی نکات ابتدایی ولی بسیار مهم در نگارش متون فارسی است. نکات گردآوری‌شده در این فصل به‌ هیچ‌ وجه کامل نیست، ولی دربردارنده‌ی حداقل مواردی است که رعایت آن‌ها در نگارش پایان‌نامه ضروری به نظر می‌رسد.


\subsection{فاصله‌گذاری}
\begin{enumerate}

\item 
علامت‌های سجاوندی مانند نقطه، ویرگول، دونقطه، نقطه‌ویرگول، علامت سؤال و علامت تعجب بدون فاصله از کلمه‌ی پیشین خود نوشته می‌شوند، ولی بعد از آن‌ها باید یک فاصله‌ قرار گیرد. مانند: من، تو، او، چرا؟ وای!
\item
علامت‌های پرانتز، آکولاد، کروشه، نقل قول و نظایر آن‌ها بدون فاصله با عبارات داخل خود نوشته می‌شوند، ولی با عبارات اطراف خود یک فاصله دارند. مانند: (این عبارت) یا \{آن عبارت\}.
\item 
دو کلمه‌ی متوالی در یک جمله همواره با یک فاصله از هم جدا می‌شوند، ولی اجزای یک کلمه‌ی مرکب باید با نیم‌فاصله\footnote{«نیم‌فاصله» فاصله‌‌ای مجازی است که در عین جدا کردن اجزای یک کلمه‌ی مرکب از یک‌دیگر، آن‌ها را نزدیک به هم نگه می‌دارد. معمولا برای تولید این نوع فاصله در صفحه‌کلید‌های استاندارد از ترکیب \lr{Shift+Space} استفاده می‌شود.}‌‌
 از هم جدا شوند. مانند: کتاب درس، محبت‌آمیز، دوبخشی.
\item 
 اجزای فعل‌های مرکب با فاصله از یک‌دیگر نوشته می‌شوند، مانند: تحریر کردن، به سر آمدن.
\end{enumerate}


\subsection{شکل حروف}
\begin{enumerate}
\item 
در متون فارسی به جای حروف «ك» و «ي» عربی باید از حروف «ک» و «ی» فارسی استفاده شود. همچنین به جای اعداد عربی مانند ٥ و ٦ باید از اعداد فارسی مانند ۵ و ۶ استفاده نمود. برای این کار، توصیه می‌شود صفحه‌کلید‌ فارسی استاندارد را روی سیستم خود فعال کنید.
\item 
عبارات نقل‌قول‌شده یا مؤکد باید درون علامت نقل قولِ «» قرار گیرند، نه ''``. مانند: «کشور ایران».
\item 
کسره‌ی اضافه‌ی بعد از «ه» غیرملفوظ به صورت «ه‌ی» یا «هٔ» نوشته می‌شود. مانند: خانه‌ی علی، دنباله‌ی فیبوناچی. اگر «ه» ملفوظ باشد، نیاز به «‌ی» ندارد. مانند: فرمانده دلیر، پادشه خوبان. 
\item 
در این نوشتار ترجیح نویسنده این است که از همزه به هر شکلی استفاده نکند. 
\end{enumerate}


\subsection{جدانویسی}
\begin{enumerate}
\item 
علامت استمرار «می» توسط نیم‌فاصله از جز بعدی فعل جدا می‌شود، مانند: می‌رود، می‌توانیم.
\item 
شناسه‌های «ام»، «ای»، «ایم»، «اید» و «اند» توسط نیم‌فاصله و شناسه‌ی «است» توسط فاصله از کلمه‌ی پیش از خود جدا می‌شوند، مانند: گفته‌ام، گفته‌ای، گفته است.
\item 
علامت جمع «ها» توسط نیم‌فاصله از کلمه‌ی پیش از خود جدا می‌شود. مانند: این‌ها، کتاب‌ها.
\item 
«به» همیشه جدا از کلمه‌ی بعد از خود نوشته می‌شود، مانند: به‌ نام و به آن‌ها، مگر در مواردی که «بـ» صفت یا فعل ساخته است، مانند: بسزا، ببینم.
\item 
«به» همواره با فاصله از کلمه‌ی بعد از خود نوشته می‌شود، مگر در مواردی که «به» جزیی از یک اسم یا صفت مرکب است، مانند: تناظر یک‌به‌یک، سفر به تاریخ. 
\item 
علامت صفت برتری «تر» و علامت صفت برترین «ترین» توسط نیم‌فاصله از کلمه‌ی پیش از خود جدا می‌شوند، مانند: سنگین‌تر، مهم‌ترین. کلمات «بهتر» و «بهترین» را می‌توان از این قاعده مستثنی نمود. 
\item 
پیشوندها و پسوندهای جامد، چسبیده به کلمه‌ی پیش یا پس از خود نوشته می‌شوند، مانند: همسر، دانشکده، دانشگاه. در مواردی که خواندن کلمه دچار اشکال می‌شود، می‌توان پسوند یا پیشوند را جدا کرد، مانند: هم‌میهن، هم‌ارزی. 
\item 
ضمیرهای متصل چسبیده به کلمه‌ی پیش‌ از خود نوشته می‌شوند، مانند: کتابم، نامت، کلامشان. 
\end{enumerate}
